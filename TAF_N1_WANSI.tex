\documentclass[12pt, a4paper]{article}
\usepackage[utf8]{inputenc}
\usepackage[T1]{fontenc}
\usepackage{lmodern}
\usepackage[french]{babel}
\usepackage{geometry}
\geometry{left=2.5cm, right=2.5cm, top=2.5cm, bottom=2.5cm}
\usepackage{graphicx}
\usepackage{array}
\usepackage{booktabs}
\usepackage{multirow}
\usepackage{fancyhdr}
\usepackage{setspace}
\usepackage{titlepic}

% En-tête et pied de page
\pagestyle{fancy}
\fancyhf{}
\renewcommand{\headrulewidth}{0pt}
\fancyfoot[C]{\thepage}

\begin{document}

\begin{titlepage}
    \centering

   

    \begin{tabular}{p{0.45\linewidth} p{0.45\linewidth}}
\centering
\textbf{RÉPUBLIQUE DU CAMEROUN}\\
******\\
Paix -- Travail -- Patrie\\
******\\
\textbf{UNIVERSITÉ DE YAOUNDÉ I}\\
******\\
\textbf {École Nationale Supérieure\\
Polytechnique de Yaoundé}\\
******\\
\textbf {Département de Génie Informatique}\\
****** &
\centering
\textbf{REPUBLIC OF CAMEROON}\\
******\\
Peace -- Work -- Fatherland\\
******\\
\textbf{UNIVERSITY OF YAOUNDÉ I}\\
******\\
\textbf {National Advanced School\\
Engineering of Yaounde}\\
******\\
\textbf {Computers Engineering Department}\\
****** \\
\end{tabular}
\vspace{3 cm}

    \begin{LARGE}

    \textbf{INTRODUCTION AUX TECHNIQUES D'INVESTIGATION NUMERIQUE}
    \end{LARGE}

    \vspace{2cm}

    \textbf{TRAVAIL A FAIRE : RESOUDRE LES EXERCICES DU CHAPITRE 1  »}

    \vspace{5cm}

    \begin{tabular}{|>{\centering\arraybackslash}p{5cm}|>{\centering\arraybackslash}p{3cm}|>{\centering\arraybackslash}p{3cm}|}
        \hline
        \textbf{NOMS \& PRENOMS} & \textbf{FILIERE  } & \textbf{MATRICULE} \\
        \hline
        WANSI GILLES GILDAS & \textbf{HN-CIN-M1} & \textbf{22P037} \\
        \hline
    \end{tabular}

    \vspace{1cm}
	\begin{Large}
	Sous la supervision de Ing THIERRY MINKA
	\end{Large}
    
    \vspace{1cm}
	\begin{Large}
	Année Académique 2025/2026
	\end{Large}

\end{titlepage}

\begin{LARGE}

\section*{Partie 1 : Fondements Philosophiques et Épistémologiques}

\subsection{1. Analyse Critique du Paradoxe de la Transparence}

\subsubsection*{Dissertation : Le paradoxe de la transparence selon Byung-Chul Han}
Byung-Chul Han, dans \emph{La Société de la transparence}, identifie un paradoxe fondamental de la modernité numérique : la quête de transparence totale entre en conflit direct avec le droit à l'intimité et à l'opacité nécessaire à l'existence humaine. Selon Han, la transparence absolue ne mène pas à la vérité, mais à un « excès de positivité » qui écrase la singularité et la complexité de l'individu. Elle transforme les sujets en objets lisibles, contrôlables et prévisibles, au détriment de leur liberté et de leur autonomie. Cette injonction à la transparence est particulièrement prégnante dans les sphères numérique, politique et sociale, où tout doit être exposé, mesuré et optimisé.

\subsubsection*{Application à un cas concret : transparence gouvernementale vs vie privée}
Prenons l'exemple d'un gouvernement qui, sous couvert de lutte contre la fraude fiscale ou le terrorisme, met en place un système de surveillance massif des transactions financières et des communications en ligne. Cette transparence imposée des citoyens envers l'État peut effectivement renforcer la sécurité ou l'efficacité administrative. Mais elle s'accompagne d'une perte de vie privée, d'une autocensure et d'une méfiance généralisée. L'enquêteur numérique, dans ce contexte, peut être tiraillé entre son devoir de collaborer avec les autorités et son obligation éthique de protéger les données personnelles des individus.

\subsubsection*{Résolution inspirée de l'éthique kantienne}
Une résolution pratique inspirée de Kant consisterait à appliquer l'\textbf{impératif catégorique} : agis de telle sorte que la maxime de ton action puisse être érigée en loi universelle. Ainsi, la transparence ne doit pas être unidirectionnelle (citoyens $\rightarrow$ État), mais \textbf{réciproque} et \textbf{régulée}. Par exemple, un système où l'accès aux données citoyennes est soumis à un contrôle juridique indépendant, où les citoyens ont un droit de regard sur les données collectées, et où la finalité de la collecte est claire, limitée et légitime. L'enquêteur pourrait ainsi œuvrer dans un cadre où la transparence sert la justice, sans instrumentaliser les personnes

\end{LARGE}
\end{document}